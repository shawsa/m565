\documentclass[12pt]{article}

%***************************************************************************************************
% Math
\usepackage{fancyhdr} 
\usepackage{amsfonts}
\usepackage{amsmath}
\usepackage{amssymb}
\usepackage{amsthm}
\usepackage{dsfont}

%***************************************************************************************************
% Macros
\usepackage{calc}

%***************************************************************************************************
% Commands and Custom Variables	
\newcommand{\problem}[1]{\hspace{-4 ex} \large \textbf{Problem #1} }
\let\oldemptyset\emptyset
\let\emptyset\varnothing
\newcommand{\norm}[1]{\left\lVert#1\right\rVert}
\newcommand{\sint}{\text{s}\kern-5pt\int}
\newcommand{\powerset}{\mathcal{P}}
\renewenvironment{proof}{\hspace{-4 ex} \emph{Proof}:}{\qed}

%***************************************************************************************************
%page
\usepackage[margin=1in]{geometry}
\usepackage{setspace}
\doublespacing
\allowdisplaybreaks
\pagestyle{fancy}
\fancyhf{}
\rhead{Shaw \space \thepage}
\setlength\parindent{0pt}

%***************************************************************************************************
%Code
\usepackage{listings}
\usepackage{courier}
\lstset{
	language=Python,
	showstringspaces=false,
	formfeed=newpage,
	tabsize=4,
	commentstyle=\itshape,
	basicstyle=\ttfamily,
}

%***************************************************************************************************
%Images
\usepackage{graphicx}
\graphicspath{ {images/} }

%tikz
\usepackage[utf8]{inputenc}
\usepackage{pgfplots}

%***************************************************************************************************
%Hyperlinks
%\usepackage{hyperref}
%\hypersetup{
%	colorlinks=true,
%	linkcolor=blue,
%	filecolor=magenta,      
%	urlcolor=cyan,
%}


\begin{document}
	\thispagestyle{empty}
	
	\begin{flushright}
		Sage Shaw \\
		m565 - Fall 2017 \\
		\today
	\end{flushright}
	
{\large \textbf{HW 4}}\bigbreak

%***************************************************************************************************
\singlespacing
\problem{1} We write Lagrange's interpolation formula as 
$$ \sum\limits_{j=0}^n l_j(x)f(x_j)  $$
Show that 
\begin{align}
	\sum\limits_{j=0}^n l_j(x)x_j \equiv x
\end{align}
	
	\doublespacing
	\begin{proof}
		Consider the function $f(x)=x$. Given the sample points $\{x_j\}_{0=j}^n$, $f(x_j)=x_j$. Since $f$ is a polynomial, it is the polynomial that interpolates the points $\Big \{ \big(x_j,f(x_j) \big) \Big\}_{0=j}^n$ and thus
		$$ \sum\limits_{j=0}^n l_j(x)x_j = \sum\limits_{j=0}^n l_j(x)f(x_j) = f(x) = x$$
	\end{proof}


%***************************************************************************************************
\singlespacing
\problem{2} Let $$E_1(x) = \int_{x}^{\infty}\frac{e^{-t}}{t}dt \ \ x>0$$ 
We would like to construct a table of values over the interval $x \in [1,10]$ such that the second degree polynomial interpolation between any three adjacent points in the table will give an error less than or eaual to $10^{-8}$.

	We will use the error formula
	$$
	\vert P_2(x) - f(x) \vert \leq \frac{\vert (x-x_0)(x-x_1)(x-x_2) \vert}{3!} \max_{a\leq x \leq b} \big\vert f^{(3)}(x) \big\vert
	$$
	First we will find the maximum of the third derivative over $1 \leq x \leq 10$.
	\begin{align*}
		f^{\prime}(x) & = \frac{d}{dx} \int_{x}^{\infty}\frac{e^{-t}}{t}dt \\
		& = \frac{d}{dx} -\int_{\infty}^{x}\frac{e^{-t}}{t}dt \\
		& = -\frac{e^{-x}}{x} \\
		& = -x^{-1}e^{-x} \\
		f^{\prime\prime}(x) & = \frac{d}{dx} -x^{-1}e^{-x} \\
		& = x^{-2} e^{-x} + x^{-1}e^{-x} \\
		f^{(3)}(x) & = \frac{d}{dx} (x^{-2} e^{-x} + x^{-1}e^{-x}) \\
		& = -2x^{-3}e^{-x} - x^{-2} e^{-x} - x^{-2} e^{-x} - x^{-1}e^{-x} \\
		& = -2x^{-3}e^{-x} - 2x^{-2} e^{-x} - x^{-1}e^{-x} \\
		& = (-2x^{-3} - 2x^{-2} - x^{-1}) e^{-x}
	\end{align*}

\end{document}
