\documentclass[12pt]{article}

%***************************************************************************************************
% Math
\usepackage{fancyhdr} 
\usepackage{amsfonts}
\usepackage{amsmath}
\usepackage{amssymb}
\usepackage{amsthm}
%\usepackage{dsfont}

%***************************************************************************************************
% Macros
\usepackage{calc}

%***************************************************************************************************
% Commands and Custom Variables	
\newcommand{\problem}[1]{\hspace{-4 ex} \large \textbf{Problem #1} }
\let\oldemptyset\emptyset
\let\emptyset\varnothing
\newcommand{\norm}[1]{\left\lVert#1\right\rVert}
\newcommand{\sint}{\text{s}\kern-5pt\int}
\newcommand{\powerset}{\mathcal{P}}
\renewenvironment{proof}{\hspace{-4 ex} \emph{Proof}:}{\qed}

%***************************************************************************************************
%page
\usepackage[margin=1in]{geometry}
\usepackage{setspace}
%\doublespacing
\allowdisplaybreaks
\pagestyle{fancy}
\fancyhf{}
\rhead{Shaw \space \thepage}
\setlength\parindent{0pt}

%***************************************************************************************************
%Code
\usepackage{listings}
\usepackage{courier}
\lstset{
	language=Python,
	showstringspaces=false,
	formfeed=newpage,
	tabsize=4,
	commentstyle=\itshape,
	basicstyle=\ttfamily,
}

%***************************************************************************************************
%Images
\usepackage{graphicx}
\graphicspath{ {images/} }

%tikz
\usepackage[utf8]{inputenc}
\usepackage{pgfplots}

%***************************************************************************************************
%Hyperlinks
%\usepackage{hyperref}
%\hypersetup{
%	colorlinks=true,
%	linkcolor=blue,
%	filecolor=magenta,      
%	urlcolor=cyan,
%}


\begin{document}
	\thispagestyle{empty}
	
	\begin{flushright}
		Sage Shaw \\
		m565 - Fall 2017 \\
		\today
	\end{flushright}
	
{\large \textbf{HW 4}}\bigbreak

%***************************************************************************************************
\singlespacing
\problem{1} We write Lagrange's interpolation formula as 
$$ \sum\limits_{j=0}^n l_j(x)f(x_j)  $$
Show that 
\begin{align}
	\sum\limits_{j=0}^n l_j(x)x_j \equiv x
\end{align}
	
	\doublespacing
	\begin{proof}
		Consider the function $f(x)=x$. Given the sample points $\{x_j\}_{0=j}^n$, $f(x_j)=x_j$. Since $f$ is a polynomial, it is the polynomial that interpolates the points $\Big \{ \big(x_j,f(x_j) \big) \Big\}_{0=j}^n$ and thus
		$$ \sum\limits_{j=0}^n l_j(x)x_j = \sum\limits_{j=0}^n l_j(x)f(x_j) = f(x) = x$$
	\end{proof}


%***************************************************************************************************
\singlespacing
\problem{2} Let $$E_1(x) = \int_{x}^{\infty}\frac{e^{-t}}{t}dt \ \ x>0$$ 
We would like to construct a table of values over the interval $x \in [1,10]$ such that the second degree polynomial interpolation between any three adjacent points in the table will give an error less than or equal to $10^{-8}$.

	We will use the error formula
	$$
	\vert P_2(x) - f(x) \vert \leq \frac{\vert (x-x_0)(x-x_1)(x-x_2) \vert}{3!} \max_{a\leq x \leq b} \big\vert f^{(3)}(x) \big\vert
	$$
	First we will find the maximum of the third derivative over $1 \leq x \leq 10$.
	\begin{align*}
		f^{\prime}(x) & = \frac{d}{dx} \int_{x}^{\infty}\frac{e^{-t}}{t}dt \\
		& = \frac{d}{dx} -\int_{\infty}^{x}\frac{e^{-t}}{t}dt \\
		& = -\frac{e^{-x}}{x} \\
		& = -x^{-1}e^{-x} \\
		f^{\prime\prime}(x) & = \frac{d}{dx} -x^{-1}e^{-x} \\
		& = x^{-2} e^{-x} + x^{-1}e^{-x} \\
		f^{(3)}(x) & = \frac{d}{dx} (x^{-2} e^{-x} + x^{-1}e^{-x}) \\
		& = -2x^{-3}e^{-x} - x^{-2} e^{-x} - x^{-2} e^{-x} - x^{-1}e^{-x} \\
		& = -2x^{-3}e^{-x} - 2x^{-2} e^{-x} - x^{-1}e^{-x} \\
		& = (-2x^{-3} - 2x^{-2} - x^{-1}) e^{-x}
	\end{align*}
	This function has no local extrema except at the end points. It is clear that the maximum value of the absolute value of the third derivative is at $x=1$, thus $\max_{1\leq x \leq 10} \big\vert f^{(3)}(x) \big\vert = \tfrac{5}{e}$. \bigbreak
	
	Next we will find the maximum value of $\vert (x-x_0)(x-x_1)(x-x_2) \vert$ over our interval. Note that if $x$ is one of our data points, this product is zero. Without loss of generality we can assume that $x_0 < x < x_1$ for some ordered points $x_0 < x_1 < x_2$. Since all of our points are equally spaced, $\vert x_0 - x_1 \vert = \vert x_1 - x_2 \vert = h$ where $h$ is the step size in our table. Letting $ x-x_0 = w$ we have the following substitutions $ x_1-x = h-w, x_2 - x = 2h - w$. Now our problem is to find the maximum value of $ w(w-h)(w-2h)$ for $ 0 \leq w \leq h$. Note that each term is positive so we may drop the absolute. Distributing we have $w^3 - 3w^2h + 2wh^2$. To maximize this we find the root of the derivative $3w^2 + 6wh + 2h^2$. Applying the quadratic formula we find that our function has a maximum at 
	$$
	w = \frac{-6h \pm \sqrt{(6h)^2 - 4(3)(2h^2)}}{2(3)} = h \frac{3-\sqrt{3}}{3}
	$$
	Substituting this back in we have the maximum value
	$$
	(h \tfrac{3-\sqrt{3}}{3})^3 - 3(h \tfrac{3-\sqrt{3}}{3})^2h + 2(h \tfrac{3-\sqrt{3}}{3})h^2 = \frac{2 \sqrt{3}}{9}h^3
	$$
	Finally we get that our error is bounded by
	$$
	\vert P_2(x) - f(x) \vert \leq \frac{5\sqrt{3}}{27e}h^3
	$$
	If we would like our error to be below $10^{-8}$ we simplify $\frac{5\sqrt{3}}{27e}h^3 < 10^{-8}$ to obtain a maximum step size
	$$
	h < \sqrt[3]{\frac{27e10^{-8}}{5 \sqrt{3}}} \approx 0.00439
	$$
	
%***************************************************************************************************
\singlespacing
\problem{3}

%***************************************************************************************************
\problem{4 (a)} Create a program for interpolating data using the the barycentric formula.

	\begin{lstlisting}
def polyinterp(u, x, y, w=None):
	if w == None:
		w = baryweights(x)
	ret = np.zeros(len(u))
	for i in range(len(ret)):
		if u[i] in x:
			ret[i] = y[np.where(x==u[i])]
		else:
			weights = w /(u[i] - x)
			ret[i] = weights.dot(y)/sum(weights)
	return ret

def baryweights(x):
	w = np.ones(len(x))
	for j, xj in enumerate(x):
		for xi in x[np.arange(len(x))!=j]: 
			w[j] /= (xj - xi)
	return w
	\end{lstlisting}

%***************************************************************************************************
\problem{4 (b)}

%***************************************************************************************************
\problem{5}

%***************************************************************************************************
\problem{6(a)} Show that the Chebyshev polynomials $T_n(x)=\cos{(n \arccos{x})}$ satisfy the following properties \\
\emph{i)} $T_n(1) = 1$ \\
\emph{ii)} $T_n(-1) = (-1)^n$ \\
\emph{iii)} $T_j(x)T_k(x) = \tfrac{1}{2}[T_{j+k}(x) + T_{j-k}(x)]$ for $j>k\leq 0$. \\
\emph{iv)} $\frac{T_{n+1}^\prime(x)}{n+1} - \frac{T_{n-1}^\prime(x)}{n-1} = 2T_n(x)$ for $n=1, 2, ...$ \\

	\begin{proof}
		Clearly $T_n(1) = \cos{(n \arccos{1})} = \cos{(n*0)} = \cos{(0)} = 1$, and thus we have (\emph{i}). \\
		In a similar fashion $T_n(-1) = \cos{(n \arccos{-1})} = \cos(n\pi)$ If $n$ is even $T_n(-1) = 1$ and if $n$ is odd then $T_n(-1)=-1$. This is also the case for $(-1)^n$. Thus $T_n(-1) = (-1)^n$ and we have (\emph{ii}). \\
		
	\end{proof}

%***************************************************************************************************
\problem{6 (b)} Prove that the Chebyshev polynomials are solutions to the differential equation
	
	\begin{align}\label{p1eq2}
	(1-x^2)y^{\prime\prime}-xy^\prime+n^2y=0
	\end{align}
	
	\begin{proof}
		Note that if $y = T_n(x)$, then
		\begin{align*}
		y^\prime & = -\sin{(n \arccos{x})} \tfrac{d}{dx}(n \arccos{x}) \\
		& = -\sin{(n \arccos{x})} (-n)(1-x^2)^{-\frac{1}{2}} \\
		& = n(1-x^2)^{-\frac{1}{2}}\sin{(n \arccos{x})}  \\
		y^{\prime\prime} &= -\tfrac{n}{2}(1-x^2)^{-\frac{3}{2}}(-2x)\sin{(n \arccos{x})} + n(1-x^2)^{-\frac{1}{2}}\cos{(n \arccos{x})}(-n)(1-x^2)^{-\frac{1}{2}} \\
		&= nx(1-x^2)^{-\frac{3}{2}}\sin{(n \arccos{x})} - n^2(1-x^2)^{-1}\cos{(n\arccos{x})}
		\end{align*}
		Then 
		\begin{align*}
		(1-x^2)y^{\prime\prime} & = (1-x^2) \Big( nx(1-x^2)^{-\frac{3}{2}}\sin{(n \arccos{x})} - n^2(1-x^2)^{-1}\cos{(n\arccos{x})} \Big) \\
		& = nx(1-x^2)^{-\frac{1}{2}}\sin{(n \arccos{x})} - n^2\cos{(n\arccos{x})} \\
		-xy^\prime & = -x n(1-x^2)^{-\frac{1}{2}}\sin{(n \arccos{x})} \\
		n^2y & = n^2\cos{(n \arccos{x})}
		\end{align*}
		Clearly these terms sum to zero and we have our result $(1-x^2)y^{\prime\prime}-xy^\prime+n^2y=0$. Thus $T_n$ are solutions to (\ref{p1eq2}).
		
	\end{proof}

%***************************************************************************************************
\singlespacing
\problem{7}

%***************************************************************************************************
\singlespacing
\problem{8}
	

\end{document}
