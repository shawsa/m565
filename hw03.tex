\documentclass[12pt]{article}


% Math		****************************************************************************************
\usepackage{fancyhdr} 
\usepackage{amsfonts}
\usepackage{amsmath}
\usepackage{amsthm}
%\usepackage{dsfont}

% Macros	****************************************************************************************
\usepackage{calc}
\newcommand{\norm}[1]{\left\lVert#1\right\rVert}

% Commands and Custom Variables	********************************************************************
\newcommand{\problem}[1]{\hspace{-4 ex} \large \textbf{#1}}
\let\oldemptyset\emptyset
\let\emptyset\varnothing

%page		****************************************************************************************
\usepackage[margin=1in]{geometry}
\usepackage{setspace}
%\doublespacing
\pagestyle{fancy}
\fancyhf{}
\rhead{Shaw \space \thepage}
\setlength\parindent{0pt}

%Code		****************************************************************************************
\usepackage{listings}
\usepackage{courier}
\lstset{
	language=Python,
	showstringspaces=false,
	formfeed=newpage,
	tabsize=4,
	commentstyle=\itshape,
	basicstyle=\ttfamily,
}

%Images		****************************************************************************************
\usepackage{graphicx}
\graphicspath{ {images/} }

%Hyperlinks	****************************************************************************************
%\usepackage{hyperref}
%\hypersetup{
%	colorlinks=true,
%	linkcolor=blue,
%	filecolor=magenta,      
%	urlcolor=cyan,
%}


\begin{document}
	\thispagestyle{empty}
	
	\begin{flushright}
		Sage Shaw \\
		m565 - Fall 2017 \\
		\today
	\end{flushright}
	
{\large \textbf{HW 3}}\bigbreak

\singlespacing
\problem{1.} Show that 
	$$
	%L^{-1} = 
	\begin{bmatrix}
		1\\
		0 & 1\\
		0 & 0 & \ddots\\
		0 & 0 & 0 & 1 \\
		0 & 0 & 0 & l_{i+1,i} & 1 \\
		0 & 0 & 0 & l_{i+2,i} & 0 & \ddots \\
		\vdots & \vdots & \vdots& \vdots& \vdots & \ddots & \ddots\\
		\vdots & \vdots & \vdots& \vdots& \vdots& \vdots & \ddots & \ddots\\
		0 & 0 & 0  & l_{n,i}  & 0 & \vdots & \vdots & \ddots & 1
	\end{bmatrix}^{-1}
	=
	\begin{bmatrix}
	1\\
	0 & 1\\
	0 & 0 & \ddots\\
	0 & 0 & 0 & 1 \\
	0 & 0 & 0 & -l_{i+1,i} & 1 \\
	0 & 0 & 0 & -l_{i+2,i} & 0 & \ddots \\
	\vdots & \vdots & \vdots& \vdots& \vdots & \ddots & \ddots\\
	\vdots & \vdots & \vdots& \vdots& \vdots& \vdots & \ddots & \ddots\\
	0 & 0 & 0  & -l_{n,i}  & 0 & \vdots & \vdots & \ddots & 1
	\end{bmatrix}
	%= L^\prime
	$$
	for $i = 1, 2, ..., n-1$.

	\begin{proof}Let the matrix on the left be $L^{-1}$ and the matrix on the right be $L^\prime$. \\
		Note that for a given $i$ the elements of the lower triangular matrix $L$ follow this form
		$$
		L_{j,k} =
		\begin{cases}
			1 & \text{if } j=k \\
			0 & \text{if } j \neq k \neq i \\
			l_{j,k} & \text{if } j > i \text{ and } k = i
		\end{cases}
		$$
		and the elements of the lower triangular matrix $L^\prime$ follow this form
		$$
		L_{j,k} =
		\begin{cases}
		1 & \text{if } j=k \\
		0 & \text{if } j \neq k \neq i \\
		-l_{j,k} & \text{if } j > i \text{ and } k = i
		\end{cases}
		$$
		Their product (also a lower triangular matrix) would then be given by
		$$
		(LL^\prime)_{j,k} = \sum\limits_{m=1}^n L_{j,m}L^\prime_{m,k}
		$$
		It remains to show that
		$$
		(LL^\prime)_{j,k} =
		\begin{cases}
			1 & \text{if } j=k \\
			0 & \text{else}
		\end{cases}
		$$
		We can now divide this problem into the following nine cases
		\begin{align*}
			\text{ (1) } j<i, k < i && \text{ (2) } j<i, k=i && \text{ (3) } j<i, k>i \\
			\text{ (4) } j=i, k < i && \text{ (5) } j=i, k=i && \text{ (6) } j=i, k>i \\
			\text{ (7) } j>i, k < i && \text{ (8) } j>i, k=i && \text{ (9) } j>i, k>i
		\end{align*}
		First note that in cases (2), (3), and (6) note that $j<k$ and we know that $(LL^\prime)_{j,k} = 0$ since the product of lower-triangular matrices is lower-triangular.
		\textbf{Case (1):} $j,k<i$
		\begin{align*}
			(LL^\prime)_{j,k=j} & = \sum\limits_{m=1}^n L_{j,m}L^\prime_{m,k} \\
			& = \sum\limits_{m=1}^{j-1} L_{j,m}L^\prime_{m,j} + L_{j,j}L^\prime_{j,j} + \sum\limits_{m=j+1}^n L_{j,m}L^\prime_{m,j} \\
			& = \sum\limits_{m=1}^{j-1} (0)(0) + (1)(1) + \sum\limits_{m=j+1}^n (0)(0) \\
			& = 1
		\end{align*}
		and if $j<k$ then $(LL^\prime)_{j,k} = 0$ as above. Lastly if $k < j$ then
		\begin{align*}
			(LL^\prime)_{j,k} & = \sum\limits_{m=1}^n L_{j,m}L^\prime_{m,k} \\
			& = \sum\limits_{m=1}^{k-1} L_{j,m}L^\prime_{m,j} + L_{j,k}L^\prime_{k,k} + \sum\limits_{m=k+1}^{j-1} L_{j,m}L^\prime_{m,j} + L_{j,j}L^\prime_{j,k} + \sum\limits_{m=j+1}^{n} L_{j,m}L^\prime_{m,j} \\
			& = \sum\limits_{m=1}^{k-1} (0)(0) + (0)(1) + \sum\limits_{m=k+1}^{j-1} (0)(0) + (1)(0) + \sum\limits_{m=j+1}^{n} (0)(0) \\
			& = 0
		\end{align*}
		\textbf{Case (4):} $k<j=i$
		\begin{align*}
			(LL^\prime)_{j,k} & = \sum\limits_{m=1}^n L_{j,m}L^\prime_{m,k} \\
			& = \sum\limits_{m=1}^{k-1} L_{j,m}L^\prime_{m,j} + L_{j,k}L^\prime_{k,k} + \sum\limits_{m=k+1}^{j-1} L_{j,m}L^\prime_{m,j} + L_{j,j}L^\prime_{j,k} + \sum\limits_{m=j+1}^{n} L_{j,m}L^\prime_{m,j} \\
			& = \sum\limits_{m=1}^{k-1} (0)(0) + (0)(1) + \sum\limits_{m=k+1}^{j-1} (0)(0) + (1)(0) + \sum\limits_{m=j+1}^{n} (0)(0) \\
			& = 0
		\end{align*}
		\textbf{Case (5):} $j=i=k$
		\begin{align*}
			(LL^\prime)_{i,i} & = \sum\limits_{m=1}^n L_{i,m}L^\prime_{m,i} \\
			& = \sum\limits_{m=1}^{i-1} L_{i,m}L^\prime_{m,i} + L_{i,i}L^\prime_{i,i} + \sum\limits_{m=i+1}^{n} L_{i,m}L^\prime_{m,i} \\
			& = \sum\limits_{m=1}^{i-1} (0)(0) + (1)(1) + \sum\limits_{m=i+1}^{n} (0)(-l_{m,i}) \\
			& = 0
		\end{align*}
		\textbf{Case (7):} $k < i < j$
		\begin{align*}
			(LL^\prime)_{j,k} & = \sum\limits_{m=1}^n L_{j,m}L^\prime_{m,k} \\
			& = \sum\limits_{m=1}^{k-1} L_{j,m}L^\prime_{m,k} + L_{j,k}L^\prime_{k,k} + \sum\limits_{m=k+1}^{i-1} L_{j,m}L^\prime_{m,k} + L_{j,i}L^\prime_{i,k} \\
				&\text{\space \space \space} + \sum\limits_{m=i+1}^{j-1} L_{j,m}L^\prime_{m,k} + L_{j,j}L^\prime_{j,k} + \sum\limits_{m=j+1}^{n} L_{j,m}L^\prime_{m,k} \\
			& = \sum\limits_{m=1}^{k-1} (0)(0) + (0)(1) + \sum\limits_{m=k+1}^{i-1} (0)(0) + (l_{j,i})(0) \\
				&\text{\space \space \space} + \sum\limits_{m=i+1}^{j-1} (0)(0) + (1)(0) + \sum\limits_{m=j+1}^{n} (0)(0) \\
			& = 0
		\end{align*}
		\textbf{Case (8):} $i=k<j$
		\begin{align*}
			(LL^\prime)_{j,i} & = \sum\limits_{m=1}^n L_{j,m}L^\prime_{m,i} \\
			& = \sum\limits_{m=1}^{i-1} L_{j,m}L^\prime_{m,i} + L_{j,i}L^\prime_{i,i} + \sum\limits_{m=i+1}^{j-1} L_{j,m}L^\prime_{m,i} + L_{j,j}L^\prime_{j,i} + \sum\limits_{m=j+1}^{n} L_{j,m}L^\prime_{m,i} \\
			& = \sum\limits_{m=1}^{i-1} (0)(0) + (l_{j,i})(1) + \sum\limits_{m=i+1}^{j-1} (0)(-l_{m,i}) + (1)(-l_{j,i}) + \sum\limits_{m=j+1}^{n} (0)(-l_{m,i}) \\
			& = l_{j,i} - l_{j,i} \\
			& = 0
		\end{align*}
		\textbf{Case (9):} $i< j,k$ \\
		If $j<k$ then $(LL^\prime)_{j,k} = 0$ as above, and if $k=j$ then
		\begin{align*}
			(LL^\prime)_{j,k=j} & = \sum\limits_{m=1}^n L_{j,m}L^\prime_{m,k} \\
			& = \sum\limits_{m=1}^{i-1} L_{j,m}L^\prime_{m,j} + L_{j,i}L^\prime_{i,j} + \sum\limits_{m=i+1}^{j-1} L_{j,m}L^\prime_{m,j} + L_{j,j}L^\prime_{j,j} + \sum\limits_{m=j+1}^{n} L_{j,m}L^\prime_{m,j} \\
			& = \sum\limits_{m=1}^{i-1} (0)(0) + (l_{j,i})(0) + \sum\limits_{m=i+1}^{j-1} (0)(0) + (1)(1) + \sum\limits_{m=j+1}^{n} (0)(0)\\
			& = 1
		\end{align*}
		And finally if $k < j$ then
		\begin{align*}
			(LL^\prime)_{j,k} & = \sum\limits_{m=1}^n L_{j,m}L^\prime_{m,k} \\
			& = \sum\limits_{m=1}^{i-1} L_{j,m}L^\prime_{m,k} + L_{j,i}L^\prime_{i,k} + \sum\limits_{m=i+1}^{k-1} L_{j,m}L^\prime_{m,k} + L_{j,k}L^\prime_{k,k} \\
				&\text{\space \space \space} + \sum\limits_{m=k+1}^{j-1} L_{j,m}L^\prime_{m,k} + L_{j,j}L^\prime_{j,k} + \sum\limits_{m=j+1}^{n} L_{j,m}L^\prime_{m,k} \\
			& = \sum\limits_{m=1}^{i-1} (0)(0) + (l_{j,i})(0) + \sum\limits_{m=i+1}^{k-1} (0)(0) + (0)(1) \\
				&\text{\space \space \space} + \sum\limits_{m=k+1}^{j-1} (0)(0) + (1)(0) + \sum\limits_{m=j+1}^{n} (0)(0) \\
			& = 0
		\end{align*}
		We have now proven that $LL^\prime = I$ and thus $L^{-1}=L^\prime$.		
	\end{proof}

\problem{2.} Illustrate how to efficiently solve a linear system of the shape below in the 9-by-9 case and determine the operation count for the general $n$-by-$n$ case.
	$$
	\begin{bmatrix}
		\text{\textbullet}&\text{\textbullet}&\text{\textbullet}&\text{\textbullet}&\text{\textbullet}&\text{\textbullet}&\text{\textbullet}&\text{\textbullet}&\text{\textbullet}\\
		\text{\textbullet}&\text{\textbullet}&&&&&&\text{\textbullet}&\text{\textbullet}\\
		\text{\textbullet}&&\text{\textbullet}&&&&\text{\textbullet}&&\text{\textbullet}\\
		\text{\textbullet}&&&\text{\textbullet}&&\text{\textbullet}&&&\text{\textbullet}\\
		\text{\textbullet}&&&&\text{\textbullet}&&&&\text{\textbullet}\\
		\text{\textbullet}&&&\text{\textbullet}&&\text{\textbullet}&&&\text{\textbullet}\\
		\text{\textbullet}&&\text{\textbullet}&&&&\text{\textbullet}&&\text{\textbullet}\\
		\text{\textbullet}&\text{\textbullet}&&&&&&\text{\textbullet}&\text{\textbullet}\\
		\text{\textbullet}&\text{\textbullet}&\text{\textbullet}&\text{\textbullet}&\text{\textbullet}&\text{\textbullet}&\text{\textbullet}&\text{\textbullet}&\text{\textbullet}\\
	\end{bmatrix}\begin{bmatrix}\\ \\ \\ \\ x \\ \\ \\ \\ \\\end{bmatrix}= \begin{bmatrix}\\ \\ \\ \\ b \\ \\ \\ \\ \\\end{bmatrix}
	$$

	Systems of this shape can be solved efficiently by a variation on Gaussian elimination. We must be careful when we generalize from the 9-by-9 case to even cases, as they will not have a middle row. This difference in shape will lead to different operation counts for odd values of $n$ and even values of $n$. Let $A$ denote our $n$-by-$n$ matrix above. \bigbreak
	
	\emph{Step 1}: For  $j=2$ through $\tfrac{n-1}{2}$ ($\tfrac{n}{2}$ for the even case) use row $j$ to eliminate the $A_{n-j, j}$ element. In counting operations, note that there are four elements in each of these rows. The multiplier must be calculated by a division. One of the elements in the row gets assigned the value $0$. The other three will involve a multiplication and a subtraction, as well as the $b$ vector. This gives $1+6+2=9$ operations. The eliminated values are marked by an $\times$ below.
	$$
	\begin{bmatrix}
	\text{\textbullet}&\text{\textbullet}&\text{\textbullet}&\text{\textbullet}&\text{\textbullet}&\text{\textbullet}&\text{\textbullet}&\text{\textbullet}&\text{\textbullet}\\
	\text{\textbullet}&\text{\textbullet}&&&&&&\text{\textbullet}&\text{\textbullet}\\
	\text{\textbullet}&&\text{\textbullet}&&&&\text{\textbullet}&&\text{\textbullet}\\
	\text{\textbullet}&&&\text{\textbullet}&&\text{\textbullet}&&&\text{\textbullet}\\
	\text{\textbullet}&&&&\text{\textbullet}&&&&\text{\textbullet}\\
	\text{\textbullet}&&&\times&&\text{\textbullet}&&&\text{\textbullet}\\
	\text{\textbullet}&&\times&&&&\text{\textbullet}&&\text{\textbullet}\\
	\text{\textbullet}&\times&&&&&&\text{\textbullet}&\text{\textbullet}\\
	\text{\textbullet}&\text{\textbullet}&\text{\textbullet}&\text{\textbullet}&\text{\textbullet}&\text{\textbullet}&\text{\textbullet}&\text{\textbullet}&\text{\textbullet}\\
	\end{bmatrix}\begin{bmatrix}\\ \\ \\ \\ x \\ \\ \\ \\ \\\end{bmatrix}= \begin{bmatrix}\\ \\ \\ \\ b \\ \\ \\ \\ \\\end{bmatrix}
	$$
	\emph{Step 2}: For $j=2$ through $\tfrac{n-1}{2}$ ($\tfrac{n}{2}$ even) use row $n-j$ to eliminate element $A_{j,n-j}$. There are three values to update, one of which is zero. So similarly to before we have $1+4+2=7$ operations.
	$$
	\begin{bmatrix}
	\text{\textbullet}&\text{\textbullet}&\text{\textbullet}&\text{\textbullet}&\text{\textbullet}&\text{\textbullet}&\text{\textbullet}&\text{\textbullet}&\text{\textbullet}\\
	\text{\textbullet}&\text{\textbullet}&&&&&&\times&\text{\textbullet}\\
	\text{\textbullet}&&\text{\textbullet}&&&&\times&&\text{\textbullet}\\
	\text{\textbullet}&&&\text{\textbullet}&&\times&&&\text{\textbullet}\\
	\text{\textbullet}&&&&\text{\textbullet}&&&&\text{\textbullet}\\
	\text{\textbullet}&&&&&\text{\textbullet}&&&\text{\textbullet}\\
	\text{\textbullet}&&&&&&\text{\textbullet}&&\text{\textbullet}\\
	\text{\textbullet}&&&&&&&\text{\textbullet}&\text{\textbullet}\\
	\text{\textbullet}&\text{\textbullet}&\text{\textbullet}&\text{\textbullet}&\text{\textbullet}&\text{\textbullet}&\text{\textbullet}&\text{\textbullet}&\text{\textbullet}\\
	\end{bmatrix}\begin{bmatrix}\\ \\ \\ \\ x \\ \\ \\ \\ \\\end{bmatrix}= \begin{bmatrix}\\ \\ \\ \\ b \\ \\ \\ \\ \\\end{bmatrix}
	$$
	\emph{Step 3}: For $j=2$ through $n-1$ use row $j$ to eliminate the $A_{1,j}$ and $A_{n,j}$ elements. Each row has three elements so as in the previous step we have $7$ operations per row. 
	$$
	\begin{bmatrix}
	\text{\textbullet}&\times&\times&\times&\times&\times&\times&\times&\text{\textbullet}\\
	\text{\textbullet}&\text{\textbullet}&&&&&&&\text{\textbullet}\\
	\text{\textbullet}&&\text{\textbullet}&&&&&&\text{\textbullet}\\
	\text{\textbullet}&&&\text{\textbullet}&&&&&\text{\textbullet}\\
	\text{\textbullet}&&&&\text{\textbullet}&&&&\text{\textbullet}\\
	\text{\textbullet}&&&&&\text{\textbullet}&&&\text{\textbullet}\\
	\text{\textbullet}&&&&&&\text{\textbullet}&&\text{\textbullet}\\
	\text{\textbullet}&&&&&&&\text{\textbullet}&\text{\textbullet}\\
	\text{\textbullet}&\times&\times&\times&\times&\times&\times&\times&\text{\textbullet}\\
	\end{bmatrix}\begin{bmatrix}\\ \\ \\ \\ x \\ \\ \\ \\ \\\end{bmatrix}= \begin{bmatrix}\\ \\ \\ \\ b \\ \\ \\ \\ \\\end{bmatrix}
	$$
	\emph{Step 4}: For $j=2$ through $n$ we use the first row to eliminate the $A_{j,1}$ element. This takes $5$ operations per row.
	$$
	\begin{bmatrix}
	\text{\textbullet}&&&&&&&&\text{\textbullet}\\
	\times&\text{\textbullet}&&&&&&&\text{\textbullet}\\
	\times&&\text{\textbullet}&&&&&&\text{\textbullet}\\
	\times&&&\text{\textbullet}&&&&&\text{\textbullet}\\
	\times&&&&\text{\textbullet}&&&&\text{\textbullet}\\
	\times&&&&&\text{\textbullet}&&&\text{\textbullet}\\
	\times&&&&&&\text{\textbullet}&&\text{\textbullet}\\
	\times&&&&&&&\text{\textbullet}&\text{\textbullet}\\
	\times&&&&&&&&\text{\textbullet}\\
	\end{bmatrix}\begin{bmatrix}\\ \\ \\ \\ x \\ \\ \\ \\ \\\end{bmatrix}= \begin{bmatrix}\\ \\ \\ \\ b \\ \\ \\ \\ \\\end{bmatrix}
	$$
	\emph{Step 5}: We can now use backward substitution to find the values of $x$. First note that $x_n=\tfrac{b_n}{A_{n,n}}$ taking one operation.
	\emph{Step 6}: Next for $j=1$ through $n-1$ we can compute $x_j=\tfrac{b_j-A_{j,n}x_n}{A_{j,j}}$, taking 3 operations per entry. This solves the system of equations. \bigbreak
	
	The final operation count for the general $n$-by-$n$ case is totaled in the table below.
	\begin{center}
		\begin{tabular}{|c|c|c|c|c|}\hline
			&\multicolumn{2}{|c|}{$n$-odd}&\multicolumn{2}{|c|}{$n$-even} \\ \hline
			\emph{Step}&FLOP count&times&FLOP count&times \\ \hline
			1 &9 & $\tfrac{n-1}{2}-1$ &9 & $\tfrac{n}{2}-1$ \\ \hline
			2 &7 & $\tfrac{n-1}{2}-1$ &7 & $\tfrac{n}{2}-1$ \\ \hline
			3 &7 & $2(n-2)$ &7 & $2(n-2)$ \\ \hline
			4 &5 & $n-1$ &5 & $n-1$ \\ \hline
			5 &1 & 1 &1 & 1 \\ \hline
			6 &3 & $n-1$ &3 & $n-1$ \\ \hline
		\end{tabular}
	\end{center}
	Finally we have our flop counts. There are $30n-59$ operations for the case when $n$ is odd, and $30n-51$ when $n$ is even.	\bigbreak
	
\problem{3.} Let $T$ be a (diagonally dominant) tridiagonal matrix, $A$ be a symmetric positive definite matrix, and $B$ and $C$ be full nonsingular matrices. Assume all of these matrices are of size $n$-by-$n$. Let $f(x)$ be defined as follows $$f(x) = x^TB^{-1}CT^{-1}A^{-1}x + b^TB^{-T}x$$ where $x$ and $b$ are column vectors of size $n$.

\problem{3. (a)} Describe how to efficiently evaluate the function $f(x)$. \\

	Since calculating the inverse of matrices is inefficient, we would like to avoid it and instead use variations on Gaussian elimination. First note that $B^{-1}$ appears twice, so factoring it once will reduce computation time. Since $A$ is SPD, it's faster to calculate the Cholesky decomposition and perform forward and backward substitution, than it is to perform Gaussian elimination. Since $T$ is banded, we will be sure to use a factorization to take advantage of this property. \bigbreak
	
	Calculate the LU factorization of $B = LU$. \\
	Note that $B^{-T}x = (LU)^{-T}x = (U^TL^T)^{-1}x = L^{-T}U^{-T}x$. \\
	Let $y = (U^T)^{-1}x$. \\
	Solve for $y$ using forward substitution on the system $U^Ty=x$ since $U^T$ is lower-triangular. \\
	Now let $w = (L^T)^{-1}y$. Solve for $w$ using backward substitution on the system $L^Tw=y$ since $L^T$ is upper-triangular. \\
	Calculate the scalar $K_2 = x^Tw$. Note that
	\begin{align*}
		K_2 = x^Tw = x^T(L^T)^{-1}y = x^T(L^T)^{-1}(U^T)^{-1}x = x^TB^{-T}x
	\end{align*}
	Calculate the Cholesky decomposition $A = HH^T$ where $H$ is a lower-triangular matrix. \\
	Calculate the LU factorization $T = EF$ taking advantage of the shape of $T$ to do this efficiently. \\
	Then $B^{-1} = U^{-1}L^{-1}$, $A^{-1}=(H^T)^{-1}H^{-1}$, and $T^{-1} = F^{-1}E^{-1}$. Substituting we get
	$$
	x^TB^{-1}CT^{-1}A^{-1}x = x^TU^{-1}L^{-1}CF^{-1}E^{-1}(H^T)^{-1}H^{-1}x
	$$
	Solve $Hy_1=x$ using forward substitution.\\
	Solve $H^Ty_2 = y_1$ using back substitution.\\
	Solve $E y_3 = y_2$ using back substitution. \\
	Solve $F y_4 = y_3$ using forward substitution. \\
	Calculate $y_5 = Cy_4$ by matrix-vector multiplication. \\
	Solve $L y_6 = y_5$ using forward substitution. \\
	Solve $U y_7 = y_6$ using back substitution. \\
	Calculate $K_1 = x^Ty_7$. \\
	It is easily shown in the manner above that $K_1 = x^TB^{-1}CT^{-1}A^{-1}x$.\\
	At last $f(x) = K_1 + K_2$.\\

\problem{3. (b)} Implement the algorithm \\
	\begin{lstlisting}
def foo_3b(x,v,A,B,C,T):
	n = A.shape[0]
	P, L, U = lu(B)
	k2 = solve_triangular(U.transpose(), x, lower=True)
	k2 = solve_triangular(L.transpose(), k2, lower=False)
	k2 = P.dot(k2)
	k2 = x.transpose().dot(k2)
	H = cholesky(A, lower=True)
	k1 = solve_triangular(H,x, lower=True)
	k1 = solve_triangular(H.transpose(),k1, lower=False)
	#convert T to a banded matrix datatype
	bandedT = np.zeros( (3,T.shape[0]) )
	bandedT[0][1:n] = T.diagonal(1)
	bandedT[1][0:n] = T.diagonal(0)
	bandedT[2][0:n-1] = T.diagonal(-1)
	k1 = solve_banded( (1,1), bandedT, k1 )
	k1 = C.dot(k1)
	k1 = P.transpose().dot(k1)
	k1 = solve_triangular(L, k1, lower=True)
	k1 = solve_triangular(U, k1, lower=False)
	k1 = v.transpose().dot(k1)
	return k1 + k2
	\end{lstlisting}
	
\problem{3. (c)} Test the code. \\
	\textbf{Test how?}\\
	With the benefit of the helper function to generate appropriate random values we can test the code above.
	\begin{lstlisting}
def foo_3_helper(n):
	A = np.random.rand(n,n)
	A = A.dot(A.transpose()) + np.identity(n)
	B = np.random.rand(n,n)
	C = np.random.rand(n,n)
	T = np.diag(np.random.rand(n-1), -1) 
	T += np.diag(np.random.rand(n), 0) 
	T += np.diag(np.random.rand(n-1), 1)
	x = np.random.rand(n,1)
	v = np.random.rand(n,1)
	return x,v,A,B,C,T
		
def foo_3b_bad(x,v,A,B,C,T):
	#used for testing only
	k2 = inv(B.transpose()).dot(x)
	k2 = x.transpose().dot(k2)
	k1 = inv(A).dot(x)
	k1 = inv(T).dot(k1)
	k1 = C.dot(k1)
	k1 = inv(B).dot(k1)
	k1 = v.transpose().dot(k1)
	return k1 + k2
	
def p3c ():
	for i in range(20):
		x,v,A,B,C,T = foo_3_helper(100)
		print(foo_3b(x,v,A,B,C,T))
	\end{lstlisting}
	Testing concludes that the calculated values are nearly the same (floating point rounding errors) and that our implementation is indeed faster for sufficiently large matricies.
	
\problem{4. (a)} Derive a fast algorithm for solving penta-diagonal systems. \\

	As in the case for tri-diagonal systems we will first find a fast algorithm for factoring, and then find fast algorithms for forward and backward substitution. In the case of the factorization, it is easy to see that the LU factorization of a penta diagonal system will be a lower triangular matrix $L$, which has zeros below the second sub-diagonal, and an upper-triangular matrix U which has zeros above the second super-diagonal. Furthermore, we can assume that the primary diagonal of one of the matrices is all ones. We will choose $U$ to have ones along the diagonal. We will also assume that the second sub diagonal of $L$ has the same values as the second subdiagonal of our penta-diagonal matrix $P$. In short, our matrix can be factored in the form below: \\
	$P=LU$ where \\
	$$P = \begin{bmatrix}
		a_1 & c_1 & e_1\\
		b_2 & a_2 & c_2 & e_2\\
		d_3 & b_3 & a_3 & c_3 & e_3\\
		 & \ddots & \ddots & \ddots  & \ddots & \ddots\\
		 &  & d_{n-2} & b_{n-2} & a_{n-2} & c_{n-2} & e_{n-2}\\
		 &  & & d_{n-1} & b_{n-1} & a_{n-1} & c_{n-1} \\
		 &  &  &  & d_{n} & b_{n} & a_{n} \\
	\end{bmatrix}$$
	$$L = \begin{bmatrix}
		l_1 \\
		k_2 & l_2 \\
		d_3 & k_3 & l_3 \\
		& \ddots & \ddots & \ddots \\
		&  & d_{n-2} & k_{n-2} & l_{n-2} \\
		&  & & d_{n-1} & k_{n-1} & l_{n-1}\\
		&  &  &  & d_{n} & k_{n} & l_{n} \\
	\end{bmatrix} \\
	$$
	$$U = \begin{bmatrix}
	1 & u_1 & w_1\\
	& 1 & u_2 & w_2\\
	&  & 1 & u_3 & w_3\\
	&  &  & \ddots  & \ddots & \ddots\\
	&  &  &  & 1 & u_{n-2} & w_{n-2}\\
	&  & &  &  & 1 & u_{n-1} \\
	&  &  &  &  &  & 1 \\
	\end{bmatrix}$$
	Given this assumption we can calculate the unknown quantities $l_j, k_j, u_j$ and $w_j$ as follows. \\
	\emph{Step 1:} 
	\begin{align*}
		l_1 &= a_1 \\
		u_1 &= c_1/l_1 \\
		w_1 & = e_1/l_1
	\end{align*}
	\emph{Step 2:} 
	\begin{align*}
		k_2 &= b_2 \\
		l_2 &= a_2 - k_2u_1 \\
		u_2 &= (c_2 - k_2w_1)/l_2 \\
		w_2 & = e_2/l_2
	\end{align*}
	\emph{Step $j$:} 
	\begin{align*}
		k_j &= b_j - d_ju_{j-2} \\
		l_j &= a_j - d_jw_{j-2} - k_ju_{j-1} \\
		u_j &= (c_j - k_jw_{j-1})/l_j \\
		w_j & = e_j/l_j
	\end{align*}
	\emph{Step $n-1$:} 
	\begin{align*}
		k_{n-1} &= b_{n-1} - d_{n-1}u_{n-3} \\
		l_{n-1} &= a_{n-1} - d_{n-1}w_{n-3} - k_{n-1}u_{n-2} \\
		u_{n-1} &= (c_{n-1} - k_{n-1}w_{n-2})/l_{n-1} \\
	\end{align*}
	\emph{Step $n$:} 
	\begin{align*}
		k_n &= b_n - d_nu_{n-2} \\
		l_n &= a_n - d_nw_{n-2} - k_nu_{n-1} \\
	\end{align*}
	
	Once factored, we now have the system $LU\vec{x}=\vec{v}$. We will substitue $U\vec{x}=\vec{y}$ so that we may first solve $L\vec{y}=\vec{v}$. We can do this forward substitution very quickly as follows\\
	\emph{Step 1:} $y_1 = v_1/l_1$ \\
	\emph{Step 2:} $y_2 = (v_2-k_2y_1)/l_2$ \\
	\emph{Step $j$:} $y_j = (v_j - d_jy_{j-2} - k_jy_{j-1})/l_j$.
	Finally we do back substitution on the system $U\vec{x}= \vec{y}$ to obtain our solution.\\
	\emph{Step $n$:} $x_n = y_n$ \\
	\emph{Step $n-1$:} $x_{n-1} = y_{n-1} - u_{n-2}x_{n-1}$ \\
	\emph{Step $j$:} $x_j = y_j - u_jx_{j+1} - w_jx_{j+2}$. \\
	This will solve the system of equations in $\mathcal{O}(n)$ time.
	
\problem{4. (b)} Determine the exact number of operations your algorithm requires.

	In the tables below we can see the flop coutns at each step and the number of times the step is performed
	\begin{center}
		\begin{tabular}{|c|c|c|}\hline
			\multicolumn{3}{|c|}{LU Factorization} \\ \hline
			\emph{Step}&FLOP count&times \\ \hline
			1 &2 & 1 \\ \hline
			2 &6 & 1 \\ \hline
			$j$ & 10 & $n-4$ \\ \hline
			$n-1$ & 9 & 1 \\ \hline
			$n$ & 6 & 1 \\ \hline
		\end{tabular}
	\end{center}
	\begin{center}
		\begin{tabular}{|c|c|c|}\hline
			\multicolumn{3}{|c|}{Forward Substitution} \\ \hline
			\emph{Step}&FLOP count&times \\ \hline
			1 &1 & 1 \\ \hline
			2 &3 & 1 \\ \hline
			$j$ & 5 & $n-2$ \\ \hline
		\end{tabular}
	\end{center}
	\begin{center}
		\begin{tabular}{|c|c|c|}\hline
			\multicolumn{3}{|c|}{Backward Substitution} \\ \hline
			\emph{Step}&FLOP count&times \\ \hline
			$n$ &0 & 1 \\ \hline
			$n-1$ &2 & 1 \\ \hline
			$j$ & 4 & $n-2$ \\ \hline
		\end{tabular}
	\end{center}
	This gives us $19n-29$ operations for the factorization (note that this is only an accurate count for $n \geq 3$ since this that is the minimum for a penta-diagonal system).
	
\problem{4. (c)} Implement your algorithm. \\

	\begin{lstlisting}
def penta_solve(P,v):
	n = P.shape[0]
	a = P.diagonal(0)
	b = np.zeros(n)
	b[1:n] = P.diagonal(-1)
	c = P.diagonal(1)
	d = np.zeros(n)
	d[2:n] = P.diagonal(-2)
	e = P.diagonal(2)
	l = np.zeros(n)
	u = np.zeros(n-1)
	w = np.zeros(n-2)
	k = np.zeros(n)
	#step 1
	l[0] = a[0]
	u[0] = c[0]/l[0]
	w[0] = e[0]/l[0]
	#step 2
	k[1] = b[1]
	l[1] = a[1]-k[1]*u[0]
	u[1] = (c[1]-b[1]*w[0])/l[1]
	w[1] = e[1]/l[1]
	#step j
	for j in range(2,n-2):
		k[j] = b[j] - d[j]*u[j-2]
		l[j] = a[j] - d[j]*w[j-2] - k[j]*u[j-1]
		u[j] = (c[j]-k[j]*w[j-1])/l[j]
		w[j] = e[j]/l[j]
	#step n-1
	k[n-2] = b[n-2] - d[n-2]*u[n-2-2]
	l[n-2] = a[n-2] - d[n-2]*w[n-2-2] - k[n-2]*u[n-2-1]
	u[n-2] = (c[n-2]-k[n-2]*w[n-2-1])/l[n-2]
	#step n
	k[n-1] = b[n-1] - d[n-1]*u[n-1-2]
	l[n-1] = a[n-1] - d[n-1]*w[n-1-2] - k[n-1]*u[n-1-1]
	#forward sub L
	y = np.zeros(n)
	#step 1
	y[0] = v[0]/l[0]
	#step 2
	y[1] = (v[1]-k[1]*y[0])/l[1]
	#step j
	for j in range(2,n):
		y[j] = (v[j] - d[j]*y[j-2] - k[j]*y[j-1])/l[j]
	#back sub U
	x = np.zeros(n)
	#step n
	x[n-1] = y[n-1]
	#step n-1
	x[n-2] = y[n-2] - u[n-2]*x[n-1]
	#step j
	for j in range(n-3, -1, -1):
		x[j] = y[j] - u[j]*x[j+1] - w[j]*x[j+2]
	return x
	\end{lstlisting}
	
\problem{4. (d)} Test your code from part (c) on the matrices specified on the HW. \\
	
	\begin{lstlisting}
def p4d_helper(n):
	a = [i for i in range(1,n+1)]
	b = [-(i+1)/3 for i in range(1, n)]
	d = [-(i+2)/6 for i in range(1,n-1)]
	P = np.zeros( (n,n) )
	P += np.diag(a, 0)
	P += np.diag(b, 1)
	P += np.diag(b, -1)
	P += np.diag(d, 2)
	P += np.diag(d, -2)
	f = np.zeros( (n,1) )
	f[0] = .5
	f[1] = 1/6.0
	f[n-2] = 1/6.0
	f[n-1] = .5
	return P, f

def p4d():
	for n in [100, 1000]:
		P,f = p4d_helper(n)
		x = penta_solve(P,f)
		r = P.dot(x)-f
		print('residual %f' % r.transpose().dot(r))
		dif = x - solve(P,f)
		print('GE %f' % dif.transpose().dot(dif))

#output
>>> p4d()
residual 0.000000
GE 0.000000
residual 0.000000
GE 0.000000
	\end{lstlisting}
	Regular Gaussian elimination gives the same solution as our algorithm, thus we can conclude that it is correct.
	
	
\problem{5. (a)} Does a small residual imply that the vector is close to the true solution of the system? \\

	No. The error on $\hat{x}$ can be understood through the following relationship 
	$$
	\frac{\norm{x - \hat{x}}}{\norm{x}} \leq \kappa(A)\frac{\norm{A\hat{x}-b}}{\norm{b}}
	$$
	If the condition number is large enough, the residual can be significantly larger than the error.
	
\problem{5. (b)} Use MATLAB to obtain an accurate solution to the given system.

	The following Python code generates an accurate solution and computes the condition number.
	\begin{lstlisting}
def p5b():
	A = np.zeros((5,5))
	A[0] = [1/2, 1/3, 1/4, 1/5, 1/6]
	A[1] = [1/3, 1/4, 1/5, 1/6, 1/7]
	A[2] = [1/4, 1/5, 1/6, 1/7, 1/8]
	A[3] = [1/5, 1/6, 1/7, 1/8, 1/9]
	A[4] = [1/6, 1/7, 1/8, 1/9, 1/10]
	b = np.array([.882, .744, .618, .521, .447]).transpose()
	x = solve(A, b)
	print(x)
	K_A = np.linalg.cond(A)
	print('Condition number %f' % K_A)
	xh = np.array([-2.1333, 0.6258, 17.4552, -11.8692, -1.4994]).transpose()
	print('norm of left %f' % (np.linalg.norm(x-xh)/np.linalg.norm(x)) )
	print('norm of right %f ' % (K_A * np.linalg.norm(b - A.dot(xh))/np.linalg.norm(b)) )
	return x, np.linalg.cond(A)
	\end{lstlisting}
	An accurate solution is $x = [-2.52, 5.04, 2.52, 7.56, -10.08]^T$. \bigbreak
	
\problem{5. (c)} Use MATLAB again to obtain the condition number for $A$. Use the appropriate result on perturbations of the RHS of a linear system to confirm that this very small residual indeed big enough to allow for the solution to be as far away from the correct one as occurs in this example. \bigbreak

	From the code in part (b) we have the condition number of $A$ is \\ 
	$\kappa(A)=1,535,043.895329$ (this is using the 2-norm).
	
\problem{6 (a)} Add code to the gepp.m function to compute the growth factor.

	MATLAB code:
	\begin{lstlisting}[language=matlab]
function [A,p,q] = lucp(A)
n = size(A,1);
p = (1:n)';
q = (1:n);
for k=1:n-1
% Find the entry that contains the largest entry in magnitude
[temp, col_max] = max(abs(A(k:n,k:n)));
[temp, x_pos] = max(col_max);
y_pos = col_max(x_pos);
row2swap = k-1+x_pos
col2swap = k-1+y_pos
% Swap the rows and columns of A and p (perform complete pivoting)
A([row2swap, k],:) = A([k, row2swap],:);
p([row2swap, k]) = p([k, row2swap]);
A(:,[col2swap,k]) = A(:,[k,col2swap]);
q(:,[col2swap,k]) = q(:,[k,col2swap]);
% Perform the kth step of Gaussian elimination
J = k+1:n;
A(J,k) = A(J,k)/A(k,k);
A(J,J) = A(J,J) - A(J,k)*A(k,J);
end
end
	\end{lstlisting}
	
\problem{6 (b)} Compute the growth factor for the matrix given.
	
	\begin{lstlisting}[language=matlab]
B = eye(21,21) - tril(ones(21,21),-1)
B(:,21)=1
[A,p,g] = lupp_growth(B)
	\end{lstlisting}
	
	The growth factor for $A$ is $g_21(A)=1,048,576 = 2^{20}$ as expected.
	
\problem{6 (c)} Show that the growth for the general $n$-by-$n$ case is exactly $g_n(A)=2^{n-1}$.

	\begin{proof}
		We will denote the matrix $A$ at the $i-1$th step of Gaussian elimination as $A^{(i)}$. Then $A^{(1)} = A$. 
		Note that $\max\limits_{1\leq i,j\leq n}\lvert A_{i,j}\rvert = 1$. For the first step in Gaussian elimination, the multiplier for each row is easily seen to be $1$. Also the first row of $A$ is zeros except $A_{1,1}=1$ and $A_{1,n}=1$. Thus the first step of Gaussian elimination will only change the first and second element of each row after the first. For the first step of Gaussian elimination, the multiplier for each row is $1$ since for $ 2 \leq j \leq n$ $A_{j,1}=-1$. Thus $A^{(2)}_{j,n} = A_{j,n} + 1A_{j,1} = 2$. Of course $A^{(2)}_{j,1}=0$ as well. Since $A_{1,i}=1$ for $2 \leq i \leq n-1$ all other entries of $A^{(2)}$ remain the same. Thus $\max\limits_{1\leq i,j\leq n}\lvert A^{(2)}_{i,j}\rvert = 2$. Suppose that after the $k$th step of Gaussian elimination we know that for $k < j \leq n$ that $A^{(k)}_{j,n}= 2^{k-1}$. Suppose also that the lower triangle of $A^{(k)}$ is the same as $A$ except for the zeros introduced by Gaussian elimination. Further suppose that for $k < i < n$ that $A^{(k)}_{j,i}= 0$. Then $\max\limits_{1\leq i,j\leq n}\lvert A^{(k)}_{i,j}\rvert = 2^{k-1}$. It is again easy to see that the multiplier for each row is $1$ and for each row $j$ where $j$ where $k < j \leq n$, the only elements that change are $A^{(k+1)}_{j,k}= 0$ and 
		\begin{align*}
			A^{(k+1)}_{j,n} &= A^{(k)}_{j,n} + 1A^{(k)}_{k,n} \\
			& = 2^{k-1} + 2^{k-1} \\
			& = 2^{k} \\
			& = 2^{(k+1)-1}
		\end{align*}
		Thus $\max\limits_{1\leq i,j\leq n}\lvert A^{(k+1)}_{i,j}\rvert = 2^{(k+1)-1}$. By induction $\max\limits_{1\leq i,j\leq n}\lvert A^{(k)}_{i,j}\rvert = 2^{k-1}$ for $1 \leq k \leq n$. Then $\max\limits_{1\leq k,i,j\leq n}\lvert A^{(k)}_{i,j}\rvert = \max\limits_{1\leq k\leq n}2^{k-1} = 2^{n-1}$. \\
		Thus $g_n(A)=2^{n-1}$.		
	\end{proof}

\problem{7 (a)} Modify the code on the course webpage to do Gaussian elimination with complete pivoting.

	This is accomplished via two separate MATLAB functions: the LU decomposition with complete pivoting, and then another file to corectly use forward and backward substitution.
	\begin{lstlisting}[language=matlab]
function [A,p,q] = lucp(A)
n = size(A,1);
p = (1:n)';
q = (1:n);
for k=1:n-1
% Find the entry that contains the largest entry in magnitude
[temp, col_max] = max(abs(A(k:n,k:n)));
[temp, x_pos] = max(col_max);
y_pos = col_max(x_pos);
row2swap = k-1+x_pos
col2swap = k-1+y_pos
% Swap the rows and columns of A and p (perform complete pivoting)
A([row2swap, k],:) = A([k, row2swap],:);
p([row2swap, k]) = p([k, row2swap]);
A(:,[col2swap,k]) = A(:,[k,col2swap]);
q(:,[col2swap,k]) = q(:,[k,col2swap]);
% Perform the kth step of Gaussian elimination
J = k+1:n;
A(J,k) = A(J,k)/A(k,k);
A(J,J) = A(J,J) - A(J,k)*A(k,J);
end
end


	\end{lstlisting}

\end{document}
