\documentclass[12pt]{article}

%%%%%%%%%%%%%%%%%%%%%%%%%%%%%%%%%%%%%%%%%%%%%%%%%%%%%%%%%%%%%%%%%%%%%%%%%%%%%%%%%%%%%%%%%%%%%%%%%%%%
% Math
\usepackage{fancyhdr} 
\usepackage{amsfonts}
\usepackage{amsmath}
\usepackage{amssymb}
\usepackage{amsthm}
%\usepackage{dsfont}

%%%%%%%%%%%%%%%%%%%%%%%%%%%%%%%%%%%%%%%%%%%%%%%%%%%%%%%%%%%%%%%%%%%%%%%%%%%%%%%%%%%%%%%%%%%%%%%%%%%%
% Macros
\usepackage{calc}

%%%%%%%%%%%%%%%%%%%%%%%%%%%%%%%%%%%%%%%%%%%%%%%%%%%%%%%%%%%%%%%%%%%%%%%%%%%%%%%%%%%%%%%%%%%%%%%%%%%%
% Commands and Custom Variables	
\newcommand{\problem}[1]{\hspace{-4 ex} \large \textbf{Problem #1} }
\let\oldemptyset\emptyset
\let\emptyset\varnothing
\newcommand{\norm}[1]{\left\lVert#1\right\rVert}
\newcommand{\sint}{\text{s}\kern-5pt\int}
\newcommand{\powerset}{\mathcal{P}}
\renewenvironment{proof}{\hspace{-4 ex} \emph{Proof}:}{\qed}
\newcommand{\RR}{\mathbb{R}}
\newcommand{\NN}{\mathbb{N}}
\newcommand{\QQ}{\mathbb{Q}}
\newcommand{\ZZ}{\mathbb{Z}}
\newcommand{\CC}{\mathbb{C}}


%%%%%%%%%%%%%%%%%%%%%%%%%%%%%%%%%%%%%%%%%%%%%%%%%%%%%%%%%%%%%%%%%%%%%%%%%%%%%%%%%%%%%%%%%%%%%%%%%%%%
%page
\usepackage[margin=1in]{geometry}
\usepackage{setspace}
%\doublespacing
\allowdisplaybreaks
\pagestyle{fancy}
\fancyhf{}
\rhead{Shaw \space \thepage}
\setlength\parindent{0pt}

%%%%%%%%%%%%%%%%%%%%%%%%%%%%%%%%%%%%%%%%%%%%%%%%%%%%%%%%%%%%%%%%%%%%%%%%%%%%%%%%%%%%%%%%%%%%%%%%%%%%
%Code
\usepackage{listings}
\usepackage{courier}
\lstset{
	language=Python,
	showstringspaces=false,
	formfeed=newpage,
	tabsize=4,
	commentstyle=\itshape,
	basicstyle=\ttfamily,
}

%%%%%%%%%%%%%%%%%%%%%%%%%%%%%%%%%%%%%%%%%%%%%%%%%%%%%%%%%%%%%%%%%%%%%%%%%%%%%%%%%%%%%%%%%%%%%%%%%%%%
%Images
\usepackage{graphicx}
\graphicspath{ {images/} }
\usepackage{float}

%tikz
\usepackage[utf8]{inputenc}
\usepackage{pgfplots}
\usepgfplotslibrary{groupplots}

%%%%%%%%%%%%%%%%%%%%%%%%%%%%%%%%%%%%%%%%%%%%%%%%%%%%%%%%%%%%%%%%%%%%%%%%%%%%%%%%%%%%%%%%%%%%%%%%%%%%
%Hyperlinks
%\usepackage{hyperref}
%\hypersetup{
%	colorlinks=true,
%	linkcolor=blue,
%	filecolor=magenta,      
%	urlcolor=cyan,
%}

\begin{document}
	\thispagestyle{empty}
	
	\begin{flushright}
		Sage Shaw \\
		m565 - Fall 2017 \\
		\today
	\end{flushright}
	
{\large \textbf{HW 6}}\bigbreak

%%%%%%%%%%%%%%%%%%%%%%%%%%%%%%%%%%%%%%%%%%%%%%%%%%%%%%%%%%%%%%%%%%%%%%%%%%%%%%%%%%%%%%%%%%%%%%%%%%%%
\problem{1} \\

	\begin{lstlisting}
def p1():
	cond = []
	est = []
	for n in range(5,20):
		H = la.hilbert(n)
		cond. append(np.linalg.cond(H))
		est.append( (1+np.sqrt(2))**(4*n) / np.sqrt(n) )
	latex_table((range(5,len(cond)+5), cond, est), 
('$n$','$K(H_n)$', 'est'))
	\end{lstlisting}
	
	\begin{center}
		\begin{tabular}{|c|c|c|c|}
			\hline
			$n$&$K(H_n)$&est&$K(H_n)/$est\\ \hline
			5&943656.0&20231528.9406&0.0466428416146\\ \hline
			6&29070279.0029&627394667.207&0.0463349156797\\ \hline
			7&985194889.72&19731957412.5&0.049928898037\\ \hline
			8&33872790819.5&627013566048.0&0.0540224209709\\ \hline
			9&1.0996509917e+12&2.00818360643e+13&0.0547584886253\\ \hline
			10&3.53537245538e+13&6.47183465942e+14&0.0546270515461\\ \hline
			11&1.23036993831e+15&2.0962052883e+16&0.0586951070667\\ \hline
			12&3.79832012269e+16&6.81776885619e+17&0.055712069488\\ \hline
%			13&4.27595335327e+17&2.22517392121e+19&0.0192162658052\\ \hline
%			14&5.96157646394e+18&7.28407419914e+20&0.00818439832017\\ \hline
%			15&8.02903661878e+17&2.3905371144e+22&3.358674739e-05\\ \hline
%			16&2.31162423425e+18&7.86292024016e+23&2.93990548504e-06\\ \hline
%			17&1.17425190356e+18&2.59132653547e+25&4.53147022379e-08\\ \hline
%			18&4.9270474355e+18&8.55486364525e+26&5.75935238691e-09\\ \hline
%			19&4.92025056332e+18&2.82862441224e+28&1.73944993971e-10\\ \hline
		\end{tabular}
	\end{center}

	Since the ratio between the actual condition number and the estimate is approximately constant we can say that this estimate effectivly describes the growth of the condition number of the Hilbert matrices. 
	
%%%%%%%%%%%%%%%%%%%%%%%%%%%%%%%%%%%%%%%%%%%%%%%%%%%%%%%%%%%%%%%%%%%%%%%%%%%%%%%%%%%%%%%%%%%%%%%%%%%%
\problem{2 (a)} \\
	
	Given the basis $ \langle 1, x, x^2 \rangle$ and the inner product $\langle f, g \rangle = \int_{-1}^1 f(x)g(x)dx$, we can create an orthogonal basis using the Gram-Schmidt Orthogonalization Process as follows:\bigbreak
	
	Let $\phi_0(x) = 1$. \\
	We would like $\phi_1 = x + a \phi_0$ so that $\phi_0$ and $\phi_1$ are orthogonal. So
	\begin{align*}
		0 & = \langle \phi_0, \phi_1 \rangle \\
		0 & = \int_{-1}^1 x + a dx \\
		0 & = \tfrac{1}{2}x^2 + ax \big\vert_{-1}^1 \\
		0 & = 2a \\
		a & = 0
	\end{align*}
	Thus $\phi_1(x) = x$. \\
	Then $\langle \phi_1, x^2 \rangle = \int_{-1}^1 x^3 dx = 0$. Also $ \langle \phi_1, \phi_1 \rangle = \int_{-1}^1 x^2 =  \tfrac{2}{3}$. Similarly $ \langle \phi_0, x^2 \rangle = \int_{-1}^1 x^2 =  \tfrac{2}{3}$. \\
	Define 
	\begin{align*}
		\phi_2(x) & = x^2 - \frac{\langle \phi_1, x^2 \rangle}{\langle \phi_1, \phi_1 \rangle} \phi_1 - \frac{\langle \phi_0, x^2 \rangle}{\langle \phi_0, \phi_0 \rangle} \phi_0 \\
		& = x^2 - 0 \phi_1 - \tfrac{1}{3}\phi_0 \\
		& = x^2 - \tfrac{1}{3}
	\end{align*}
	The new functions $\phi_0, \phi_1, \phi_2$ are an orthogonal basis of the space generated by $1, x$ and $x^2$.

%%%%%%%%%%%%%%%%%%%%%%%%%%%%%%%%%%%%%%%%%%%%%%%%%%%%%%%%%%%%%%%%%%%%%%%%%%%%%%%%%%%%%%%%%%%%%%%%%%%%	
\problem{2 (b)} \\

	Using the recurrance relation
	\begin{align}
		\phi_{n+1} - (x - \beta) \phi_n + \gamma_n \phi_{n-1} = 0 \label{eq1}
	\end{align}
	Show that 
	\begin{align*}
		\beta_n & = - \frac{\langle x \phi_n, \phi_n \rangle}{\norm{\phi_n}^2} & &\text{\ and} & \gamma_n = \frac{\norm{\phi_n}^2}{ \norm{\phi_{n-1}}}
	\end{align*}
	
	\begin{proof}
		From (\ref{eq1}) we have that
		\begin{align*}
			\phi_{n+1} & = (x + \beta_n) \phi_n - \gamma_n \phi_{n-1} \\
			\phi_n \phi_{n+1} & = (x + \beta_b)\phi_n \phi_n - \gamma_n \phi_{n-1}\phi_n \\
			\langle \phi_n, \phi_{n+1} \rangle & = \langle \phi_n, \phi_nx \rangle + \beta_b \langle \phi_n, \phi_n \rangle - \gamma_n \langle \phi_{n-1}, \phi_n \rangle \\
			0 & = \langle \phi_n, \phi_nx \rangle + \beta_b \norm{\phi_n}^2 \\ 
			\beta_n & = - \frac{\langle x \phi_n, \phi_n \rangle}{\norm{\phi_n}^2}
		\end{align*}
		Also from (\ref{eq1}) we have that
		\begin{align}
			\phi_{n+1} & = (x + \beta_n) \phi_n - \gamma_n \phi_{n-1} \nonumber \\
			\phi_{n+1} \phi_{n+1} & = (x + \beta_b)\phi_{n+1} \phi_n - \gamma_n \phi_{n-1}\phi_{n+1} \nonumber \\
			\langle \phi_{n+1}, \phi_{n+1} \rangle & = \langle \phi_{n+1}, \phi_nx \rangle + \beta_b \langle \phi_n, \phi_{n+1} \rangle - \gamma_n \langle \phi_{n-1}, \phi_{n+1} \rangle \nonumber \\
			\norm{\phi_{n+1}}^2 & = \langle \phi_{n+1}, \phi_nx \rangle \label{eq2}
		\end{align}
		Lastly from (\ref{eq1}) we have
		\begin{align*}
			\phi_{n+1} & = (x + \beta_n) \phi_n - \gamma_n \phi_{n-1} \\
			\phi_{n-1} \phi_{n+1} & = (x + \beta_b)\phi_n \phi_{n-1} - \gamma_n \phi_{n-1}\phi_{n-1} \\
			\langle \phi_{n-1}, \phi_{n+1} \rangle & = \langle \phi_{n}, \phi_{n-1}x \rangle + \beta_b \langle \phi_{n-1}, \phi_n \rangle - \gamma_n \langle \phi_{n-1}, \phi_{n-1} \rangle \\
			\gamma_n \norm{\phi_{n-1}}^2 & = \langle \phi_n, \phi_{n-1}x \rangle \\ 
			\gamma_n \norm{\phi_{n-1}}^2 & = \norm{\phi_{n}}^2 \text{\ \ \ \ \  (by (\ref{eq2}))}\\ 
			\gamma_n & = \frac{\norm{\phi_n}^2}{ \norm{\phi_{n-1}}}
		\end{align*}
	\end{proof}

%%%%%%%%%%%%%%%%%%%%%%%%%%%%%%%%%%%%%%%%%%%%%%%%%%%%%%%%%%%%%%%%%%%%%%%%%%%%%%%%%%%%%%%%%%%%%%%%%%%%
\problem{2 (c)} \\

	Let $\phi_{-1}(x) = 0$ and $\phi_0(x) = 1$. Then 
	\begin{align*}
		\beta_0 & = - \frac{\langle x \phi_0, \phi_0 \rangle}{\norm{\phi_0}^2} \\
		& = - \frac{\langle x, 1 \rangle}{\norm{1}^2} \\
		& = 0 \\
	\end{align*}
	Since $\beta_0 = 0$ and $\phi_0 = 0$ we have from the recurrence relation that $\phi_1 = x$. Then
	\begin{align*}
	\beta_1 & = - \frac{\langle x \phi_1, \phi_1 \rangle}{\norm{\phi_1}^2} \\
		& = - \frac{\langle x^2, x \rangle}{\norm{x}^2} \\
		& = 0 \\
		\gamma_1 & = \frac{\norm{\phi_1}^2}{ \norm{\phi_{0}}} \\
		& = \frac{\tfrac{2}{3}}{1} = \tfrac{1}{3}
	\end{align*}
	Thus $\phi_2 = x^2 - \tfrac{1}{3}$.
		
\end{document}
